% \documentclass[a4paper, 10px]{ctexart}
\documentclass{ctexart}
\usepackage{amsmath}
\usepackage[left=1in, right=1in, top=1.2in, bottom=1.2in]{geometry}
\usepackage{ctex}
\usepackage[utf8]{inputenc}
\usepackage{boxproof}
\usepackage{fontspec}
\usepackage{a4wide}
\usepackage{listings}
% \setmainfont[Scale = 1]{Palatino}
% \setCJKmainfont{Songti SC}
\usepackage{fancyhdr}
\pagestyle{fancy}
\fancypagestyle{plain}{
    \fancyhead[L]{East China Normal University}
    \fancyhead[R]{}
    \fancyfoot[C]{\thepage}
}
\def\premise{\mathrm{premise}}
\def\assumption{\mathrm{assumption}}
\def\MT{\mathrm {MT\ }}
\def\LEM{\mathrm{LEM}}
\def\intro{\mathrm{i\ }}
\def\elim{\mathrm{e\ }}
\def\introa{\mathrm{i_1\ }}
\def\elima{\mathrm{e_1\ }}
\def\introb{\mathrm{i_2\ }}
\def\elimb{\mathrm{e_2\ }}



\def\n{\neg}
\def\d{\vee}
\def\c{\wedge}
\def\NNF{\texttt{NNF}}
\def\IMPLFREE{\texttt{IMPL\_FREE}}
\def\CNF{\texttt{CNF}}
\def\DISTR{\texttt{DISTR}}

\title{Logic in Computer Science Assignment 2}
\author{10185101210 陈俊潼}
\date{October 2020}

\begin{document}

\maketitle

\section{}

\subsection{Convert the following formula into CNF.}

$\neg(r \rightarrow(\neg((p \vee q) \wedge(\neg p \rightarrow(q \wedge r)))))\\$

Let $\phi = \neg(r \rightarrow(\neg((p \vee q) \wedge(\neg p \rightarrow(q \wedge r)))))$.

First apply \IMPLFREE:

$$
\begin{aligned}
    \IMPLFREE(\phi) &= \n(r\to (\n((p\d q) \c (\n p \to (q \c r))))) \\
    &= \n(\IMPLFREE(r \to (\n ((p \d q) \c  (\n p \to (q \c r)))))) \\
    &= \n(\n r \d \IMPLFREE (\n((p \d q) \c (\n p \to (q \c r))))) \\
    &= \n(\n r \d (\n(\IMPLFREE (p \d q) \c \IMPLFREE (\n p \to (q \c r))))) \\
    &= \n(\n r \d (\n((p \d q) \c (\IMPLFREE (\n p \to (q \c r)))))) \\
    &= \n(\n r \d (\n((p \d q) \c (\n \n p \d \IMPLFREE(q \c r))))) \\
    &= \n(\n r \d (\n((p \d q) \c (\n \n p \d (\IMPLFREE q \c \IMPLFREE r))))) \\
    &= \n(\n r \d (\n((p \d q) \c (\n \n p \d(q \c r))))) \\
\end{aligned}
$$

Then apply \NNF:

$$
\begin{aligned}
    \NNF(\IMPLFREE(\phi)) &= \NNF(\n(\n r \d (\n((p \d q) \c (\n \n p \d(q \c r)))))) \\
    &= \NNF(\n \n r) \c \NNF (\n (\n ((p \d q) \c (\n \n p \d (q \c r))))) \\
    &= r \c \NNF ((p \d q) \c (\n \n p \d (q \c r))) \\
    &= r \c (\NNF(p \d q) \c \NNF (\n \n p \d (q \c r))) \\
    &= r \c ((\NNF p \d \NNF q) \c (\NNF (\n \n p) \d \NNF (q \c r)))\\
    &= r \c ((p \d q) \c (p \d (q \c r)))
\end{aligned}
$$

\newpage

Then apply \CNF:

$$
\begin{aligned}
    \CNF(\NNF(\IMPLFREE(\phi))) &= \CNF(r \c ((p \d q) \c (p \d (q \c r)))) \\
    &= r \c \CNF((p \d q) \c (p \d (q \c r)))\\
    &= r \c \CNF(p \d q) \c \CNF(p \d (q \c r))\\
    &= r \c \DISTR(p, q) \c \DISTR(\CNF(p), \CNF((q \c r))) \\
    &= r \c \DISTR(p, q) \c \DISTR(p, (q \c r)) \\
    &= r \c (p \d q) \c \DISTR(p, q) \c \DISTR(p, r)\\
    &= r \c (p \d q) \c (p \d q) \c (p \d r)\\
    &= r \c (p \d q) \c (p \d r) \\
\end{aligned}
$$

Above is the result of \CNF \ algorithm. Further more, this result can be simplified by:

$$
\begin{aligned}
    r \c (p \d q) \c (p \d r) &=  r \c (p \d r) \\
\end{aligned}
$$

\subsection{Determine the satisfiablity of the following formula with Horn algorithm.}

$
(p \wedge q \wedge s \rightarrow \perp) \wedge(q \wedge s \rightarrow p) \wedge(\top \rightarrow s) \wedge(s \rightarrow q)\\
$

Marked initially: $\top$

Since $\top \to s$, mark s. (Now marked: s, $\top$)

Since $s \to q$, mark q. (Now marked: s, q, $\top$)

Since $q \c s \to p$, mark p. (Now marked: s, p, q, $\top$)

Since $p \c q \c s \to \bot$, mark bot. (Now marked: s, p, q, $\top$, $\bot$)

Since $\bot$ is marked at last, from Horn's algorithm, this formula is not satisfiable.
\end{document}
