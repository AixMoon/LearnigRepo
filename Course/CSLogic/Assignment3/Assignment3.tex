% \documentclass[a4paper, 10px]{ctexart}
\documentclass{ctexart}
\usepackage[left=1in, right=1in, top=1.2in, bottom=1.2in]{geometry}
\usepackage{ctex}
\usepackage[utf8]{inputenc}
\usepackage{boxproof}
\usepackage{fontspec}
\usepackage{a4wide}
% \setmainfont[Scale = 1]{Palatino}
% \setCJKmainfont{Songti SC}
\usepackage{fancyhdr}
\pagestyle{fancy}
\fancypagestyle{plain}{
    \fancyhead[L]{East China Normal University}
    \fancyhead[R]{}
    \fancyfoot[C]{\thepage}
}
\def\premise{\mathrm{premise}}
\def\assumption{\mathrm{assumption}}
\def\MT{\mathrm{MT\ }}
\def\LEM{\mathrm{LEM}}
\def\PBC{\mathrm{PBC\ }}
\def\intro{\mathrm{i\ }}
\def\elim{\mathrm{e\ }}
\def\introa{\mathrm{i_1\ }}
\def\elima{\mathrm{e_1\ }}
\def\introb{\mathrm{i_2\ }}
\def\elimb{\mathrm{e_2\ }}

\def\n{\neg}
\def\d{\vee}
\def\c{\wedge}
\def\a{\forall}
\def\e{\exists}

\title{Logic in Computer Science Assignment 3}
\author{10185101210 陈俊潼}
\date{November 2020}

\begin{document}

\maketitle

\section{Prove}
Prove the following theorems with deduction rules.

\subsection{$\forall x(P(x) \rightarrow \neg Q(x)) \vdash \neg(\exists x(P(x) \wedge Q(x)))$}

Proof:

$$
\begin{proofbox}
    \: \a x (P(x) \to \n Q(x)) \= \premise\\
    \[
        \: \e x(P(x) \c Q(x)) \= \assumption\\
        \[
            x_0 \: P(x_0) \c Q(x_0) \= \assumption \\
            \: P(x_0) \= \c \elima 3 \\ 
            \[
                \: Q(x_0) \= \c \elimb 3\\
                \: P(x_0) \to \n Q(x_0) \= \a \elim 1 \\
                \: \n Q(x_0) \= \to \elim 6\\
            \]
            \: \bot \= \n \intro 5,7 \\
        \]
        \: \bot \= \e \elim 2, 3-8\\
    \]
    \: \n(\e x(P(x) \c Q(x))) \= \n \intro 2,9
\end{proofbox}
$$

\newpage

\subsection{$\forall x(P(x) \leftrightarrow x=b) \vdash P(b) \wedge \forall x \forall y(P(x) \wedge P(y) \rightarrow x=y)$}

Proof: 
$$
\begin{proofbox}
    \: \a (P(x) \to x = b) \= \premise \\
    \: \a (x = b \to P(x)) \= \premise \\
    \: b = b \= = \intro \\
    \: b = b \to P(b) \= \a \elim 2\\
    \: P(b) \= \to\elim 4\\
    \[
        x_0 \: P(x_0) \to x_0 = b \= \a\elim 1\\
        \[
            y_0 \: P(y_0) \to y_0 = b \= \a\elim 1\\
            \[
                \: P(x_0) \c P(y_0) \= \assumption \\
                \: P(x_0) \= \c\elima 8 \\
                \: x_0 = b \= \to\elim 6 \\
                \: P(y_0) \= \c\elimb 8 \\
                \: y_0 = b \= \to\elim 7 \\
                \: x_0 = y_0 \= =\elim 10, 12\\
            \]
            \: P(x_0) \c P(y_0) \to x_0 = y_0 \= \to\intro 8,9-13\\
        \]
        \: \a y (P(x_0) \c P(y) \to x_0 = y) \= \a \intro 7-14\\
    \]
    \: \a x \a y (P(x) \c P(y) \to x = y) \= \a \intro 6-15\\
    \: P(b) \c \a x \a y (P(x) \c P(y) \to x = y)\ \ \ \ \  \= \c\intro 5,16
\end{proofbox}$$

\end{document}
